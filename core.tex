\documentclass[11pt, letterpaper]{article}
\title{Building the foundations of a game engine}
\author{Zachary Boehm}
\date{October 2024}

% Set teh font
\usepackage{helvet}
\renewcommand{\familydefault}{\sfdefault}

% Placeholder text
\usepackage{blindtext}

% Change the margins and geometry of the document
\usepackage{geometry}

% Remove the numbers from sections only in the document
\usepackage{titlesec}
\titleformat{\section}{\normalfont\Large\bfseries}{}{0pt}{}
\titleformat{\subsection}{\normalfont\large\bfseries}{}{0pt}{}
\titleformat{\subsubsection}{\normalfont\small\bfseries}{}{0pt}{}

% Color and link toc
\usepackage{color}   %May be necessary if you want to color links
\usepackage{hyperref}
\hypersetup{
    colorlinks=true, %set true if you want colored links
    linktocpage=true, %set to all if you want both sections and subsections linked
    %linktoc=all, % Use this for poth title and page number
    linkcolor=blue,  %choose some color if you want links to stand out
}

% Allows manipulating the line spacing
\usepackage{setspace}
\onehalfspacing

\usepackage{wrapfig} % Wrap figures/images

% Import graphics
\usepackage{graphicx}
\graphicspath{{images/}}

\begin{document}

\maketitle

\break

\tableofcontents

\break

\newgeometry{bottom=1.5cm, right=2cm, left=2cm, top=1.5cm}

\section{Prelude}

The following sections will reference a ``library''. This library is some low level core/system library that handles cross-platform interactions with the operating system. Each section will describe a part of the functionality that this library can take on as a cross-platform operating system API layer.

\section{Windowing}

Creating and managing windows in most operating systems are very similar; the program asks the OS for resource to create a window, then the program polls for events and dispatches or translate the events sent by the OS to the window.

Once the program has a window it is responsible for handling events sent from the operating system for the window. These events can include and are not limited to: input, resizing, minimizing, maximizing, focus, and much more.

\subsection{Menu, Caption, Title}
\blindtext
\subsubsection{Title/Caption}
\blindtext
\subsubsection{Custom Title/Caption}
\blindtext
\subsubsection{Menu}
\blindtext
\subsubsection{Macos Menu}
\blindtext
\subsubsection{Linux Menu}
\blindtext

\subsection{SysTray}
\blindtext
\subsubsection{What is it}
\blindtext
\subsubsection{Icon}
\blindtext
\subsubsection{Context Menu}
\blindtext

\subsection{Event Loop}
\blindtext
\subsubsection{Concept}
\blindtext
\subsubsection{Types of events}
\blindtext
\subsubsection{Handling Events}
\blindtext
\subsubsection{Updating the Affected Window}
\blindtext

\subsection{Input}
\blindtext
\subsubsection{Keyboard}
\blindtext
\subsubsection{Keyboard translation challenges}
\blindtext
\subsubsection{Mouse}
\blindtext
\subsubsection{Controller}
\blindtext

\section{Audio}
\subsection{System level interactions}
\subsection{Audio file types}
\subsection{Affects}

\section{Why Not Included Graphics?}
\subsection{Using existing technology and libraries}
\subsection{Library Bindings}

\end{document}
